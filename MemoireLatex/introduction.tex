
\chapter*{General Introduction}





\textbf{}

\addcontentsline{toc}{chapter}{\numberline{}{General Introduction}}

Systems are becoming more complex, in part due to increased customer requirements and the
expectation that applications should be seamlessly integrated with other existing, often distributed
applications and systems. In addition, there is an increasing demand for these complex systems to
exhibit some type of intelligence as well.


Multi-agent systems tend to be more robust and, in many cases, more efficient (due to their ability to perform parallel actions) than single monolithic applications. In addition, the individual agents tend to be simpler to build, as they are built from a single agents perspective.


Graphs represent a graphical and direct tool for visualizing the complex structure of a system. It is a practical and intuitive tool for modeling. And when we talk about modeling multi-agent system we have serval framawork to use OMACS , PNS  Are very practical examples of modeling graphs


OMACS  is  a framework to allow us  to define a graph represent a Multi-agent System with his component ( Agents , Roles , Capabilities , Goals ) and the relation between these components .
 
 	
PNS is also an other framework  basically is developed for chemical reaction systems , each node in this framework represent a material and between two material there is a transition they called it operating unit , but we will use it  to represent a multi agent system . 


And it have some algorithms we need in our research ,  the goal of our research is to optimize a multi agent system by applying one of these algorithms , because of that we need to propose a new approach allow to transform a OMACS graph to PNS graph  .


Our approach is base on Graph Transformation in AToM3 Tools  , and this document will show you our work and the framework I used  in three chapiter : 

\begin{enumerate}
\item Chapter 1 : First Framework OMACS
 \begin{enumerate}
 	\item Introduction
    \item Definition
    \item Main Element 
    \item Relation between Elements
    \item Conclusion
    
  \end{enumerate}
\item Chapter 2 : Second Framework PNS
 \begin{enumerate}
 	\item Introduction
    \item Definition of P-Graph
    \item Definition of PNS 
    \item Mathematical Definition
    \item Algorithm  
    \item Conclusion
  \end{enumerate}

\item Chapter 3 : Our Approach of transformation and the tool i will use for transformation $AToM^3$
and the steps of transformation , some tips how to use this tool and finally we use other tools to apply the algorithm of optimization which called P-Graph Sudio.

\end{enumerate}




