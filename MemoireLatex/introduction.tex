
\chapter*{General Introduction}





\textbf{}

\addcontentsline{toc}{chapter}{\numberline{}{General Introduction}}


Multiagent systems have become popular over the last few years for building complex, adaptive systems
in a distributed, heterogeneous setting. The problem is that many multiagent systems are typically designed to work within 
a limited set of configurations. Even when the system possesses the resources and
computational power to accomplish its goal, it may be constrained by its own structure and knowledge
of its members capabilities. To overcome these problems, they have developed a framework that allows
the system to design its own organization at runtime. The key component of there framework is the
Organization Model for Adaptive Computational Systems (OMACS)\cite{omacs4}.

OMACS defines the knowledge needed about a systems structure and capabilities to allow it to reorganize at runtime in the face of a changing environment and its agents capabilities\cite{omacs4}\cite{omacs2}.
 

OMACS  is  a framework to allow us  to define a graph represent a Multi-agent System with his component (Agents, Roles, Capabilities, Goals) and the relation between these components.
 
 	
PNS is also an other framework basically is developed to design a chemical reaction systems, each node in this framework represent a material and between two material there is a transition they called it operating unit, but we will use it  to represent a multi agent system. 


And it have some algorithms we need in our project, the goal of our project is to optimize a multi agent system by applying one of these algorithms, because of that we need to propose a new approach allow to transform a OMACS graph to PNS graph.


Our approach is base on Graph Transformation in AToM3 Tools, and this document will show you our work and the framework I used  in four chapiter : 

In the first chapter we presente some technologie, and framework used to  modilize multi-agent system. and the second chapter illustrate other frame work is diferent then the first, we use it also to modilize MaS.
the third chapter cover some concept of transformation, after that we represent the tools start with  $AToM3$ we use to modilize the formalism of these frameworks and the second PGraph-Studio which have the algorithm to apply on pns models, the fourth chapter  explain our approach of transformation.

finaly this document end with general conclusion which is a collection of the main idea in this document.

 



