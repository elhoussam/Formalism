\chapter*{General Introduction}

\textbf{}

\addcontentsline{toc}{chapter}{\numberline{}{General Introduction}}


Multiagent systems have become popular over the last few years for building complex, adaptive systems. 
The problem is that many multiagent systems are typically designed to work within 
a limited set of options. Even when the system possesses the resources and
computational power to accomplish its goal, it may be constrained by its own structure and knowledge
of its members capabilities. To overcome these problems, The framework OMACS is developed to allows the system, design its own organization at runtime \cite{omacs4}.

OMACS defines the knowledge needed about a systems structure and capabilities to allow it to reorganize at runtime in the face of a changing environment and its agents capabilities\cite{omacs4}\cite{omacs2}.
However, the OMACS Framework does not have a way (Method) to find an optimum organization, to overcome this problem, we include other framework named PNS.

PNS is also an other framework basically developed to design a chemical reaction systems, represented by biparte graph, each node in this Graph represent a material. Between two material there is a transition called operating unit. In our work we will use it to represent a multi agent system. 


The aim of our project is to find the best organization (optimum structure) of a multi agent system by applying one of these algorithms, we propose a new approach of model-to-model transformation. And this later allow us to transform OMACS model into PNS model. Our approach is based on graph transformation using AToM3 Tool.


Our document is organized as follows :

The first and the second chapter we presente and illustrate the OMACS and PNS frameworks. And both framework are used to model a multi-agent system.
The third chapter cover some concept of model-to-model transformation, and the tools we used to implements our approachs, The fourth chapter illustrate and explain our approach of transformation. 
Finaly this document end with general conclusion which is a collection of the main idea in this document.

 



