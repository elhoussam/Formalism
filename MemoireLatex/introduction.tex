
\chapter*{General Introduction}





\textbf{}

\addcontentsline{toc}{chapter}{\numberline{}{General Introduction}}

Systems are becoming more complex, in part due to increased customer requirements and the
expectation that applications should be seamlessly integrated with other existing, often distributed
applications and systems. In addition, there is an increasing demand for these complex systems to
exhibit some type of intelligence as well.

No longer is it \quotedblbase good enough \textquotedblright to be able to access systems across the internet,
but customers require that their systems know how to access data and
systems, even in the face of unexpected events or failures.

The goal of our research is to develop a framework for constructing complex, distributed systems
that can autonomously adapt to their environment. Multiagent systems have become popular over
the last few years for providing the basic notions that are applicable to this problem. A multiagent

system uses groups of self-directed agents working together to achieve a common goal. Such
multiagent systems are widely proposed as replacements for sophisticated, complex, and
expensive stand-alone systems for similar applications. Multiagent systems tend to be more
robust and, in many cases, more efficient (due to their ability to perform parallel actions) than
single monolithic applications. In addition, the individual agents tend to be simpler to build, as
they are built from a single agent?s perspective.

However, unpredictable application environments make multiagent systems susceptible to
individual failures that can significantly reduce the ability of the system to accomplish its goal.

The problem is that multiagent systems are typically designed to work within a limited set of
configurations. Even when the system possesses the resources and computational ability to
accomplish its goal, it may be constrained by its own structure and knowledge of its member?s
capabilities, which may change over time. 

In most multiagent design methodologies  ,the system designer analyzes the possible organizational 
structure - which determines which roles are required to accomplish which goals and sub-goals 
and then designs one organization that will suffice for most anticipated scenarios. 

Unfortunately, in dynamic applications where the environment as well as the agents 
may undergo changes, a designer can rarely account for, or even consider, all possible situations.
Attempts to overcome these problems include large-scale redundancy using homogenous 
capabilities and centralized/distributed planning. 

However,homogenous approaches negate many of the benefits of using a multiagent approach and are not
applicable in complex applications where specific capabilities are often needed by only one or
two agents. Centralized and distributed planning approaches tend to be brittle and
computationally expensive due to their required level of detail (individual actions in most cases).

To overcome these problems, we are developing a framework that allows a system to design its
own organization at runtime. In essence, we propose to provide the system with organizational
knowledge and let the system design its own organization based on the current goals and its
current capabilities. 

While the designer can provide guidance, supplying the system with key
organizational information will allow it to redesign, or reorganize, itself to match its scenario.
This paper presents a key component of our framework, a metamodel for multiagent
organizations named the Organization Model for Adaptive Computational Systems (OMACS).

OMACS defines the requisite knowledge of a system?s organizational structure and capabilities
that will allow the system to reorganize at runtime and enable it to achieve its goals in the face of
a changing environment and its agent?s capabilities.