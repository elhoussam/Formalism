\chapter*{General Introduction}

\textbf{}

\addcontentsline{toc}{chapter}{\numberline{}{General Introduction}}


Multiagent systems have become popular over the last few years for building complex, adaptive systems. 
The problem is that many multiagent systems are typically designed to work within 
a limited set of options. Even when the system possesses the resources and
computational power to accomplish its goal, it may be constrained by its own structure and knowledge
of its members capabilities. To overcome these problems, The framework OMACS is developed to allows the system, design its own organization at runtime \cite{omacs4}.

OMACS defines the knowledge needed about a systems structure and capabilities to allow it to reorganize at runtime in the face of a changing environment and its agents capabilities\cite{omacs4}\cite{omacs2}.

OMACS is a framework, that allow us to define a Multi-agent System with his component (Agents, Roles, Capabilities, Goals) and the relation between these components, 
by an oriented graph. Each component represent a node, and each relation represent an edge.
 
PNS is also an other framework basically developed to design a chemical reaction systems, represented by biparte graph, each node in this Graph represent a material and between two material there is a transition they called it operating unit, in our work we will use it to represent a multi agent system. 

PNS come with 3 Algorithm to extract information from the PNS model. and we need these algorithm in our project, because of that we add this framework 
to our work.

The aim of our project is to find the best organization, optimum structure of a multi agent system by applying one of these algorithms, 
because of that we need to propose a new approach of model-to-model transformation, lead us to PNS model from a OMACS model. Our approach is base on Graph Transformation in AToM3 Tool.

This document will show you our work and the framework I used in four chapiter : 

The first and the second chapter we presente, illustrate the previous frameworks, and we use it to modeling multi-agent system. and there methodes, relations and algorithms.
the third chapter cover some concept of model-to-model transformation, then we represente the tools AToM3, we use to modelize the formalism of these frameworks and the second PGraph-Studio which have the algorithm to apply on pns models. 
The fourth chapter explain our approach of transformation. 
Finaly this document end with general conclusion which is a collection of the main idea in this document.

 



