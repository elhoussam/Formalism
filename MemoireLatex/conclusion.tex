
\chapter*{General Conclusion}

\addcontentsline{toc}{chapter}{\numberline{}{Conclusion Générale}}

Through this paper, we saw out objective was to optimise a multi agent system 
and this MaS represented according the OMACS framework .

We have introduce  in the first chapter the definition of elements and the relation between these element in OMACS .

The Second chapter was the  PGraph Framework and  Process Network Synthesis , the basic concept and mathematical definition  of this framework , and i mention 
three algorithm applicable on PNS 

in the Third chapter , we presented our approach for transforam MaS in OMACS presentation into PNS Model , we start with defining the meta model for both framework i use ( OMACS and PNS )  followed by an overview about the Tool $AToM^3$ that tool allow us to define the meta model and generate a formalism 
from these meta model , and use formalism to create ot modelise you own model  .

the next step it is to define a graph grammar that allows to transform from source graph in this case is OMACS model into target graph (PNS model) , after execute this transforamtion we can not be used directly for optimisation with the existing tools , for thsi we have to propose another approach to transforam 
the model in PNS into xml file .

Now we can import the xml file into the PGraph-Studio and apply the Algorithm 
of optimisation . 
